\documentclass[openany]{scrbook}

\usepackage[utf8]{inputenc}
\usepackage[]{amsmath}
\usepackage[]{amssymb}
\usepackage[]{amsfonts}
\usepackage[]{hyperref}
\hypersetup{
    colorlinks=true,
    linkcolor=blue,
    filecolor=magenta,      
    urlcolor=cyan,
}

\usepackage{titling}    % gives you \thedate \theauthor  \thetitle as maketitle removes them

\usepackage{blindtext}
\newcommand{\filler}{\blindtext[1]}

\newcommand{\matr}[1]{\mathbf{#1}} % undergraduate algebra version
%\newcommand{\matr}[1]{#1}          % pure math version
%\newcommand{\matr}[1]{\bm{#1}}     % ISO complying version

\newcommand{\vect}[1]{\mathbf{#1}}
%\newcommand{\vect}[1]{\vec{#1}}
%\newcommand{\vect}[1]{\boldsymbol{#1}}

\title{Mathematics}
\author{Christopher Finch}
\date{\today}

\begin{document}
\frontmatter
\maketitle
\tableofcontents

\mainmatter
\addchap{Introducton}
\filler

\chapter{Set theory}

If no elements are in common then  
\begin{equation*}
A \cap B = \varnothing \\
A \cup B \neq \varnothing
\end{equation*}.

If no elements are in common then
\begin{center}
$ A \cap B = \varnothing $ and $ A \cup B \neq \varnothing $
\end{center}

$\therefore$ since the empty set is an element of all sets (see \href{https://proofwiki.org/wiki/Empty_Set_is_Subset_of_All_Sets}{proof} that empty set is subset of all sets) then $ A\cap B \in A \cup B $.

If elements in $A$ and $B$ are the same, i.e. $ \forall a \in A : a \in B $ (which is the same as saying $A \subseteq B$) and likewise $ \forall b \in B : b \in A $ (all of this previous verbiage can be described as $A = B$) then $A \cap B = A\cap A = A $ and $A\cup B = A\cup A = A$


$ f:\mathbb N\to\mathbb N $

$n\mapsto5n$

\[A=\left\{x:x \text{ is a letter in the English alphabet, }x \text{ is a vowel} \right\}\]

Observe that $b \notin A$, $e \in A$ and $p \notin A$.

\[B=\{2,4,6,\dots \}\]


\dictum{
    By the end of this section you will be able to
    \begin{itemize}
        \item 
        \item 
    \end{itemize}

After that, we shall prove an analogous theorem in the ring of polynomials over a field.
}


\chapter{Eigenvalues and Eigenvectors}
\section{Introduction to Eigenvalues and Eigenvectors}

\dictum{By the end of this section you will be able to 
\begin{itemize}
    \item determine eigenvalues and eigenvectors
    \item prove properties of eigenvalues and eigenvectors
\end{itemize}
}
\subsection{Interesting asides}
Eigen{vectors/values} occur frequently in physics and engineering. Google's search engine relies upon eigenvalues and eigenvectors.

\subsection{Definition}
They take the form $\matr{A}\vect{v}=(\text{scalar}) \times \vect{v}$ where $\vect{v}$ is a non-zero vector and $\matr{A}$ is a square matrix.

\subsection{Practical use}
With the Matrix's Eigen values/vectors you can
\begin{itemize}
    \item determine the determinant of the matrix
    \item determine if the matrix is invertible
    \item determine the powers of the matrix 
\end{itemize}

\subsection{Examples}
Let $\matr{A}=()  $


\end{document}